\newpage
\section{Określenie wymagań szczegółowych}		%2
%Napisać gdzie używa się tego algorytmu
%Opisać sposób działania programu/algorytmu
%Napisać spsoób wykorzystania algorytmu po przez wykonanie przykładu (np. mnożenie macierzy - wykonać ręcznie przykład z mnożeniem macierzy pokazujący jak mnoży się macierz ręcznie)

Obecnie istnieje rosnące zapotrzebowanie na narzędzie umożliwiające efektywne monitorowanie aktywności fizycznych, tworzenie planów treningowych oraz śledzenie postępów w zdrowym stylu życia i fitness. Brakuje jednolitej aplikacji, która łączy te funkcje w spójnym i użytecznym narzędziu. Istnieje potrzeba stworzenia kompleksowej aplikacji sportowej, która spełni te wymagania.

Aplikacja ta ma na celu pomóc użytkownikom, zwłaszcza osobom aktywnie dążącym do lepszej kondycji fizycznej i zdrowego stylu życia, w zarządzaniu swoimi aktywnościami fizycznymi. Wyzwania, które chcemy rozwiązać, obejmują:

\begin{itemize}
    \item Różnorodność Aktywności Fizycznych: Użytkownicy angażują się w różne aktywności fizyczne, takie jak bieganie, joga, siłownia, tenis, rower itp. Brakuje jednego narzędzia, które pozwoli na monitorowanie wszystkich tych aktywności.

    \item Tworzenie Planów Treningowych: Ludzie potrzebują elastyczności w tworzeniu spersonalizowanych planów treningowych, które uwzględniają ich cele i poziom zaawansowania.

    \item Monitorowanie Postępów: Użytkownicy chcą mieć możliwość śledzenia swoich postępów, zarówno w zakresie wykonywanych ćwiczeń, jak i w ogólnym osiągnięciu celów fitness.

    \item Rejestracja Trasy Treningu: Dla aktywności na zewnątrz, takich jak jogging, jazda na rowerze rejestrowanie trasy treningu jest ważne dla użytkowników, aby śledzić swoje wyniki i postępy oraz mieć możliwość powrotu do przebytej trasy z przeszłości.

    \item Dziennik Treningów: Użytkownicy chcieliby mieć możliwość automatycznego zapisywania swoich treningów i dziennika aktywności.

    \item Baza Danych Ćwiczeń: W aplikacji musi istnieć szeroka baza danych ćwiczeń, co ułatwi użytkownikom znalezienie i dodanie ćwiczeń do swojego planu treningowego.
\end{itemize}

Nasza aplikacja sportowa ma na celu skupić się na tych potrzebach i zapewnić użytkownikom narzędzie, które pomoże im osiągnąć swoje cele fitness i cieszyć się zdrowym stylem życia. Analiza problemu jasno pokazuje, że istnieje zapotrzebowanie na takie rozwiązanie.

\subsection{Technologia}  %2.1

Do tworzenia naszej aplikacji sportowej wybraliśmy Xamarin.

Jest to narzędzie i platforma umożliwiająca tworzenie aplikacji mobilnych na różne platformy, takie jak Android i iOS.Dzięki temu oszczędzamy czas i zasoby, unikając konieczności pisanie dwóch osobnych aplikacji. Xamarin korzysta z języka programowania C\#, który jest silnie typowanym językiem programowania ogólnego przeznaczenia.

C\# jest używany do tworzenia logiki biznesowej i interfejsu użytkownika w Xamarin.
Jednym z głównych jego atutów jest możliwość dzielenia wspólnego kodu źródłowego między różnymi platformami, co pozwala na efektywną pracę przy jednoczesnym tworzeniu aplikacji.

Do tworzenia aplikacji można używać popularnych zintegrowanych środowisk programistycznych (IDE), takich jak Visual Studio (z dostępem do narzędzi Xamarin) lub Xamarin Studio.

Oferuje dostęp do szerokiej gamy bibliotek i narzędzi, które ułatwiają tworzenie aplikacji mobilnych, w tym interfejsów użytkownika, dostępu do urządzeń i innych funkcji.

